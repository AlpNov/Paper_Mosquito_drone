%  links for CASE 2016, http://case2016.org/  
% https://ras.papercept.net/conferences/scripts/login.pl      to submit
% https://ras.papercept.net/conferences/scripts/pdftest.pl  to test pdf

\documentclass[letterpaper, 10 pt, conference]{ieeeconf}  % Comment this line out if you need a4paper

%\documentclass[a4paper, 10pt, conference]{ieeeconf}      % Use this line for a4 paper

\IEEEoverridecommandlockouts                              % This command is only needed if 
                                                          % you want to use the \thanks command

\overrideIEEEmargins                                      % Needed to meet printer requirements.

% See the \addtolength command later in the file to balance the column lengths
% on the last page of the document

% The following packages can be found on http:\\www.ctan.org
%\usepackage{graphics} % for pdf, bitmapped graphics files
%\usepackage{epsfig} % for postscript graphics files
%\usepackage{mathptmx} % assumes new font selection scheme installed
%\usepackage{times} % assumes new font selection scheme installed
%\usepackage{amsmath} % assumes amsmath package installed
%\usepackage{amssymb}  % assumes amsmath package installed
\usepackage{overpic}
\graphicspath{{./pictures/},{./pictures/pdf/},{./pictures/ps/},{./pictures/png/},{./pictures/jpg/}}
\usepackage{amsmath}
\usepackage[table,xcdraw]{xcolor}
\newcommand{\todo}[1]{\vspace{5 mm}\par \noindent \framebox{\begin{minipage}[c]{0.98 \columnwidth} \ttfamily\flushleft \textcolor{red}{#1}\end{minipage}}\vspace{5 mm}\par}

\title{\LARGE \bf
Using Unmanned Vehicles With Instrumented Bug Zappers\\ To Detect \& Eliminate Mosquitoes
}


\author{Mary Burbage, An Nguyen, Kyle Walker, Nhan Phung, Vinh Truong, Erik Van Aller, and Aaron T. Becker% <-this % stops a space
\thanks{M. Burbage, A. Nguyen, K. Walker, N. Phung, V. Truong, E. Van Aller, and A. Becker are with the ECE Department at the University of Houston, TX.
        {\tt\small atbecker@uh.edu}}%
}

\begin{document}

\maketitle
\thispagestyle{empty}
\pagestyle{empty}


%%%%%%%%%%%%%%%%%%%%%%%%%%%%%%%%%%%%%%%%%%%%%%%%%%%%%%%%%%%%%%%%%%%%%%%%%%%%%%%%
\begin{abstract}
Mosquitoes are a vector for several deadly diseases which are responsible for killing millions of people each year.  Popular methods to control mosquitoes such as insecticides are effective, but long-term effects of pesticides are of concern, particularly as mosquito species develop resistance over time.
Traditional electrified screens (bug zappers) use UV light to attract pests but have a large by-catch of non-pest insects. This paper introduces techniques using electrified screens (bug zappers) mounted on unmanned vehicles to autonomously seek out and eliminate mosquitoes in their breeding grounds. Instrumentation on the bug zappers logs the GPS location, altitude, weather details, and time of each mosquito elimination. Mosquito control offices could use this information to analyze the insects' activities. The device can be mounted on a remote controlled or autonomous unmanned vehicle. If autonomous, the vehicle can use the data collected from the electrified net as feedback to improve the effectiveness of the motion plan. 
This paper examines design considerations, presents a working prototype system of a drone and instrumented bug zapper, and introduces a simulator for swarms of mosquitoes and a mosquito-eliminating drone.  
\end{abstract}


%%%%%%%%%%%%%%%%%%%%%%%%%%%%%%%%%%%%%%%%%%%%%%%%%%%%%%%%%%%%%%%%%%%%%%%%%%%%%%%%
\section{Introduction}

Mosquito-borne diseases kill millions of humans each year~\cite{murray2012global}. Popular methods to control mosquitoes such as insecticides are effective, but they have the potential to introduce long-term environmental damage and mosquitos have demonstrated the ability to become resistant to pesticides~\cite{ndiath2012resistance}. Traditional electrified screens (bug zappers) use UV light to attract pests but have a large bycatch of non-pest insects~\cite{University-Of-Florida1997}. This paper introduces techniques using bug zappers mounted on unmanned vehicles to autonomously seek out and eliminate mosquitoes in their breeding grounds and swarms. Instrumentation on the bug zappers logs the GPS location, altitude, weather details, and time of each mosquito hit.  Mosquito control offices can use this information to analyze the insects' activities. The device can be mounted on a remote-controlled or autonomous unmanned vehicle. If autonomous, the vehicle can use the data collected from the electrified net as feedback to improve the effectiveness of the motion plan. 

Initial work simulates a large number of mosquitoes within a rectangular area. Each mosquito obeys a biased random walk flight pattern. The flying robot can eliminate any modeled mosquito that intersects its path, detect the time each mosquito is eliminated, and use this data as feedback for a motion policy.


  \begin{figure}
\centering
\begin{overpic}[width=0.9\columnwidth]{DroneAndNet.jpg}\end{overpic}
\caption{\label{fig:DroneAndNet}
A multi-copter drone carrying a $60~cm$ square bug-zapping screen. An onboard microcontroller monitors the voltage across the screen and records the time, GPS location, altitude, and weather information for each mosquito strike.} 
\end{figure}

  
  
  \section{Related Work}
  
  \subsection{Mosquito Control Solutions}
  
	Mosquito control has a long history of efforts associated both with monitoring mosquito populations ~\cite{dennett2007associations} and with eliminating mosquitoes.  The work involves both draining potential breeding grounds and destroying living mosquitoes \cite{peter2005tick}.  An array of insecticidal compounds has been used with different application methods, concentrations, and quantities, including both larvicides and compounds directed at adult mosquitoes \cite{larvicides2005guidelines}.
	
	Various traps have been designed to capture and/or kill mosquitoes with increasing sophistication in imitating human bait as designers strive to achieve a trap that can rival the attraction of a live human \cite{maliti2015development}.  In recent history, methods have also included genetically modifying mosquitoes so that they either cannot reproduce effectively or cannot transmit diseases successfully \cite{marshall2009malaria}, and with the recent genomic mapping of mosquito species, new ideas for more targeted work have been formulated \cite{hill2005arthropod}.
 
    \subsection{Robotic Pest Management}
    
As GPS technology has flourished and data processing has become cheaper and more readily available, researchers have explored options for implementing the new technologies in breeding ground removal \cite{anupa2014identification} and more effective insecticide dispersion \cite{hur2015low}.  Low cost drones for residential spraying are under development \cite{amenyo2014medizdroids}.  Even optical solutions have been considered, including laser containment \cite{boonsri2012laser} or, by extension, exclusion and laser tracking and extermination \cite{kare2010build}.
    
    \subsection{Robotic coverage}
    Robotic coverage has a long history. The basic problem is one of designing a path for a robot that ensures the robot visits within $r$ distance of every point on the workspace.  For an overview see \cite{Choset2001}.  This work has been extended to use multiple coverage robots in a variety of ways, including using simple behaviors for the robots \cite{spears2006physics,Koenig2001}.
    The key difference in the mosquito coverage problem is that the mosquitoes can move, recontaminating an area previously cleared. We instead have a probability of coverage, as in~\cite{Das2011}.  This is closely related to the art gallery problem~\cite{lee1986computational} but with limited range of visibility.
    
  \section{Design}
  
   \subsection{Electronics}
   
Fig. \ref{fig:Block_Diagram} shows a block diagram of the control circuit.  The system is powered by a $9~V$ Lithium Polymer battery applied directly to the controller. The $9~V$ supply is also regulated down to $3~V$ and applied to the voltage multiplier circuit that powers the mesh of the net, as shown in Fig. \ref{fig:CircuitDiagram}. One or more meshes may be used to enable localization of mosquito hits. The net output is a high voltage across the screen mesh.   A protection circuit steps this voltage down to a suitable level for monitoring by the ADC of the controller. The controller uses a GPS shield for monitoring the location and altitude as well as a real time clock to timestamp each data point collected from the system.     

                \begin{figure}
\centering
\begin{overpic}[width=0.9\columnwidth]{Block_Diagram_RevC.pdf}\end{overpic}
\caption{\label{fig:Block_Diagram}
Block diagram of microcontroller and bug zapper.
} 
\end{figure}
  
A circuit diagram of both the bug zapper and the probe is shown in Fig. \ref{fig:CircuitDiagram}.  The bug zapper (shown on the left) uses a BJT (Q1) and center tap transformer to invert a DC input voltage to AC and apply it to the primary winding of a step-up transformer  (1:1000). The voltage at the secondary winding of the transformer is boosted and rectified to a high voltage output capacitor that is applied to the inner layer of the mesh.  The probe (shown on the right) uses a voltage divider to lower the voltage so that it can be monitored by the microcontroller.  A Zener diode (D4) placed in parallel protects the ADC input.
  
                \begin{figure}
\centering
\begin{overpic}[width=1.0\columnwidth]{CircuitDiagram.pdf}\end{overpic}
\caption{\label{fig:CircuitDiagram}
  Circuit diagram of the bug zapper and probe circuit.
  } 
\end{figure}
  
  The screen is constructed with two layers of wire mesh with $3.6~mm$ openings with a layer of wire mesh with $0.74~mm$ openings suspended between them.  A balsa wood frame insulates the outer meshes from the inner mesh.  This mesh arrangement weighs $1633~g/m^2$.  The payload that the drone can reliably carry is used to select a size that satisfies the requirements, $60~cm$ for this multi-coptor drone.

  \subsection{Energy Budget}
  
  Tests with an oscilloscope show that in the steady state, a $60~ cm \times 60~cm$ screen and electronics have a power consumption of $17.3~ W$.  During a zap, the screen voltage monitoring circuit shorts briefly when the mosquito contacts the screen.  The recovery time is approximately $80~ \mu s$.  Fig.~\ref{fig:BugZapTrace} shows the time sequence.  The current recovery can be modeled with an exponential fit.
 \begin{align} \label{eq:BugZapFit}
i=-69.1e^{-2.7\times10^4 t} ~A
\end{align}

The fit in \eqref{eq:BugZapFit} gives us a time constant of $2.7\times10^4~/s$.  Multiplying voltage by current to find the instantaneous power ($p=iv$) and integrating the area under the power curve show a total energy consumption of $4.2~mJ$ for each zap.  This indicates that charging the screen is the most significant use of energy for the system; killing ten thousand mosquitoes is equivalent to only $2.4~s$ of screen-charging time.
  
                \begin{figure}
\centering
\begin{overpic}[width=1.0\columnwidth]{BugZapTrace.pdf}\end{overpic}
\caption{\label{fig:BugZapTrace}
  Current, voltage, and power traces as an \textit{Aedes triseriatus} mosquito contacts the bug-zapping screen.  It causes a brief short that recovers in $80~ \mu s$.
  } 
\end{figure}
  
  \subsection{Location of screen}
 The drone carries the bug-zapping screen, which is suspended by cables at each corner.  The location of this screen determines the efficacy of the mosquito drone, measured in mosquitoes eliminated per second of flight time.

For manufacturing ease, the electrified screen is a square measuring $d_s$ on each side. The mosquito species we are initially targeting fly at low altitude, so the screen is suspended a distance $h_s$ beneath the drone flying at height $h_d$.
Suspending this screen beneath the drone improves efficiency because a hanging screen requires less weight than a rigid frame to hold the screen above the drone.  This screen can be suspended at any desired angle $\theta$ in comparison to horizontal, as shown in Fig.~\ref{fig:DroneConfigs}.
A key question is what distance $h_s$ the screen should be suspended from the drone, and the optimal angle $\theta$.  The goal is to clear the greatest volume of mosquitoes per second, a volume defined by the drone forward velocity $v_f$ and the cross sectional area $h_m \times d_s$ cleared by the screen, as shown in Fig.~\ref{fig:AngleVsSpeed}.

 To hover, the drone must push sufficient air down with velocity $v_d$ to apply a force that cancels the pull of gravity.  The drone has mass $m_{d}$ and has a square-shaped cross section of size $d_d \times d_d$.  The mass flow of air through the drone's props is equal to the product of the change in velocity of the air, the density of the air $\rho_a$, and the cross sectional area.
 
     \begin{figure}
\centering
\begin{overpic}[width=0.9\columnwidth]{DroneConfigs.pdf}\end{overpic}
\caption{\label{fig:DroneConfigs}
The drone suspends a square bug-zapping screen beneath it.  Propwash pushes incoming mosquitoes downwards, and the drone clears a volume $h_m \times d_s \times v_f$.} 
\end{figure}


We assume that air above the drone is quiescent, so the change in velocity of the air is $v_d~ m/s$.
 \begin{align} \label{eq:forceBalanceForDrone}
 \text{Force gravity} & = \left(\text{mass flow}\right) \cdot \text{air velocity} \nonumber \\
 m_{d} \cdot  g &= (v_d \cdot  \rho_a \cdot  d_d^2 ) \cdot  v_d 
% \text{kg} \cdot \frac{ \text{m}}{ \text{s}^2}&= \left( \frac{ \text{m}}{\text{s}} \cdot  \frac{ \text{kg}}{\text{m}^3}  \cdot \text{m}^2 \right) \cdot  \frac{ \text{m}}{\text{s}}\nonumber
\end{align}

Then the required propwash, the velocity of air beneath the drone, for hovering is
 \begin{align} \label{eq:dronePropwash}
v_d = \sqrt{ \frac{ m_d g}{\rho_a d_d^2} }
\end{align}
The drone testing site in Houston, Texas is 15 m above sea level. At sea level the density of air $\rho_a$ is $1.225~ kg/m^3$.
The 3DR Solo drone weighs $2~ kg$ with a diameter of $0.71~ m$~\cite{Sollenberger2015}. The acceleration due to gravity is $9.871~ m/s^2$.  Substituting these values into \eqref{eq:dronePropwash} gives $v_d = 5.6~ m/s$.

Due to propwash, a mosquito in level flight will fall relative to the drone at a rate of $v_d/v_f$.  As shown in Fig.~\ref{fig:DroneConfigs}, we can extend lines with slope $v_d/v_f$ from the screen's trailing edge to $h_{top}$ and from the leading edge to $h_{bottom}$
 \begin{align} \label{eq:ClearedCrossSection}
h_{top} &= h_d - h_s + \frac{d_s}{2} \sin(\theta) +  \frac{d_d + d_s\cos(\theta)}{2}  \frac{v_d}{v_f} \nonumber \\
h_{bottom} &= h_d - h_s - \frac{d_s}{2} \sin(\theta) +  \frac{d_d - d_s\cos(\theta)}{2}  \frac{v_d}{v_f}  \nonumber \\
h_m &= h_{top} - h_{bottom} =  d_s\left(\frac{v_d}{v_f}\cos(\theta) + \sin(\theta) \right)
\end{align}
The optimal angle is therefore a function of forward and propwash velocity:
\begin{align} \label{eq:OptimalScreenAngle}
\ \theta = \mathrm{ArcTan}\left(\frac{v_f}{v_d}\right)
\end{align}

To ensure the maximum number of mosquitoes are collected, the screen must be sufficiently far below the drone $ h_s > \frac{d_s}{2} \sin(\theta) +  \frac{d_d + d_s\cos(\theta)}{2}  \frac{v_d}{v_f}$  and the bottom of the screen must not touch the ground, $ h_d > h_s + \frac{d_s}{2} \sin(\theta) $.

      \begin{figure}
\centering
\begin{overpic}[width=0.9\columnwidth]{AngleVsSpeed.pdf}\end{overpic}
\caption{\label{fig:AngleVsSpeed}
The volume cleared by a drone is a function of screen angle $\theta$ and forward velocity $v_f$. The dotted line shows the optimal angle given in \eqref{eq:OptimalScreenAngle}. } 
\end{figure}
 
Changing the flying height $h_d$ of the drone will target different mosquito populations   because mosquitoes are not distributed uniformly vertically.  Gillies and Wilkes have demonstrated that different species of mosquitoes prefer to fly at different heights \cite{gillies1976vertical}.
  
    \section{Simulation}
    
   Before launching a fully-instrumented drone, this concept was simulated using \textsc{Matlab} code.  One thousand mosquitoes are randomly placed within a square area one hundred meters on a side.  A satellite image of Houston was used as the simulation environment, and each mosquito is programmed to move according to a biased random walk at a speed up to $0.4~ m/s$ and with a direction heading biased toward the greenest of the pixels surrounding its current position.  This imitates the live mosquitoes' preference for vegetative areas.  
   The simulation is initialized by a mosquito distribution generated by running the mosquito movement routine  five thousand times, simulating 1.4 hours of flying time, before the robot begins to search (Fig.~\ref{fig:SimulationSetupTime0}).  Because mosquitoes do not care about boundaries, a toroidal mapping is used to keep them in the workspace.  
    
Two different routines handle the robot's path.  The first uses a random bounce algorithm.  The robot begins in the center of the workspace and moves with a heading that varies randomly up to $\pm 0.2~ rad$ from its previous heading and bounces off the perimeter of the workspace with a random heading equally biased between 0 and $2\pi$, excluding headings leading outside the workspace.  The velocity is set at a prescribed speed.

The second algorithm uses a boustrophedon path \cite{Choset2001}.  The robot begins in one corner of the workspace and methodically progresses back and forth, advancing one screen width at each turn.  If it covers the entire field in the allotted time, it begins covering the field again.

To keep the routines comparable, the robots use the same speed and same number of iterations.  These simulations used $12~ m/s$ and fifteen minutes of flying time.  They also used the same image for biasing the mosquito flight.

For the main body of the simulation, a loop runs a series of iterations in which each mosquito moves one step and the robot moves one step.  In that step, the path traced by the bug-zapping screen is calculated, and any mosquitoes in that path are considered to have been killed.

        \begin{figure}
\centering
\begin{overpic}[width=0.9\columnwidth]{SimulationSetupTime0.pdf}\end{overpic}
\caption{\label{fig:SimulationSetupTime0}
A $100~m\times100~m$ image with one thousand simulated mosquitoes (red markers).  The mosquitoes have had 1.4 hours to bias themselves toward green areas of the image.  The robot (red circle) is shown at the upper left corner, preparing to start a bug-zapping run. } 
\end{figure}

%            \begin{figure}
%\centering
%\begin{overpic}[width=0.9\columnwidth]{SimulationSetupEnd.pdf}\end{overpic}
%\caption{\label{fig:SimulationSetupEnd}
%A 100$\times$100m image with the robot (red circle) and the area swept out by the bug-zapping net in a time step (red rectangle) shown along with a population of a thousand mosquitoes.  Those that were killed during the simulation are shown in white, and those that survived are shown in black. } 
%\end{figure}

One hundred trials were performed with each coverage path and the results evaluated.  The boustrophedon successfully covered the entire field in every trial ($\mu=100\%$, $\sigma=0\%$) and covered a portion of the area a second time, while the random bounce covered only 67.9\% of the field ($\mu=67.9\%$, $\sigma=1.0\%$) with a considerable amount of overlap.  As a result, the boustrophedon killed significantly more mosquitoes ($\mu=87.7\%$, $\sigma=1.1\%$) than the random bounce ($\mu=68.6\%$, $\sigma=2.7\%$).  In fact, the smallest number of mosquitoes killed in 100 trials by the boustrophedon (84.1\%) was considerably larger than the largest number killed by the random bounce (75.1\%).  Fig.~\ref{fig:BoustrophedonPath} and Fig.~\ref{fig:RandomPath} show the paths for boustrophedon and random bounce trials, respectively.

        \begin{figure}
\centering
\begin{overpic}[width=0.9\columnwidth]{BoustrophedonPath.pdf}\end{overpic}
\caption{\label{fig:BoustrophedonPath}
Boustrophedon robot path (yellow lines) overlaid upon a $100~m\times100~m$ image.  The robot (red circle) and the area covered by the bug-zapping net in an iteration (red rectangle) are shown along with a population of a thousand mosquitoes.  Those that were killed during the simulation are shown in white, and those that survived are shown in red.} 
\end{figure}

        \begin{figure}
\centering
\begin{overpic}[width=0.9\columnwidth]{RandomPath.pdf}\end{overpic}
\caption{\label{fig:RandomPath}
Random bounce robot path (yellow lines) overlaid upon a $100~m\times100~m$ image.  The robot (red circle) and the area covered by the bug-zapping net in an iteration (red rectangle) are shown along with a population of a thousand mosquitoes.  Those that were killed during the simulation are shown in white, and those that survived are shown in red. } 
\end{figure}

        \begin{figure}
\centering
\begin{overpic}[width=0.9\columnwidth]{SimulationResults.jpg}\end{overpic}
\caption{\label{fig:SimulationResults}
Comparison of percentage of area covered and percentage of mosquitoes killed for the boustrophedon and random bounce coverage patterns.  Plots show aggregate results of 100 trials, using the workspace shown in Fig.~\ref{fig:SimulationSetupTime0}.}
\end{figure}

%Data
%The files used for this run of simulations are as follows:
%MosquitoFlightSimv2p0.m
%MosquitoSimv2p2randombounce.m
%MosquitoSimv2p2boustrophedon.m
%MosquitoSimComparev1p0.m
%
%All are located in the Dropbox code folder.
%Raw data are located in the Excel file Simulation Results.xlsx.
%    
    
    \section{Experiment}
    
Before mounting an instrumented screen to a drone, a proof of concept experiment used three commercial bug zappers mounted on a pole with two controllers and a GPS tracker as shown in Fig. ~\ref{fig:BugZapperOnPole}.  The circuits were powered with a $9~ V$ battery as described above.  One controller logged the GPS data at one second intervals while the other continuously tracked the voltage across each bug zapper mesh.  Both sets of data are time-stamped so that they can be correlated to determine the location of each kill.  Fig.~\ref{fig:GPSpathAndKill} shows the ten-minute circuit around a body of water followed in the first trial with the kill locations superimposed upon it.        

            \begin{figure}
\centering
\begin{overpic}[width=0.9\columnwidth]{BugZapperOnPole.jpg}\end{overpic}
\caption{\label{fig:BugZapperOnPole}
To calibrate the system, three instrumented bug zappers were carried on a $2.5~ m$ pole.} 
\end{figure}

%        \begin{figure}
%\centering
%\begin{overpic}[width=0.9\columnwidth]{DronePathAndKills.pdf}\end{overpic}
%\caption{\label{fig:DronePathAndKills}
%    GPS path (blue line) and kill locations (red markers).} 
%\end{figure}

        \begin{figure}
\centering
\begin{overpic}[width=0.9\columnwidth]{GPSpathAndKill.jpg}\end{overpic}
\caption{\label{fig:GPSpathAndKill}
    Map overlaid with GPS path (red line) and kill locations (blue circles) from initial trial in March 2016.  Visualization using~\cite{Schneider2003}.} 
\end{figure}

\section{Conclusion and Future Work}

Initial tests indicate that we can use our instrumentation to track the location of a mosquito-killing drone as it patrols a field and that we can detect and map mosquito kills.  Simulation has shown that covering the field thoroughly leads to a higher success rate when killing mosquitoes.  Because the mosquitoes move around, the robot will not kill all the mosquitoes in a single trial as some can fly into a previously covered area as the robot approaches them; however, the chances of killing the mosquitoes increase as the coverage increases.

%Future work
There are a number of refinements to the simulation algorithm that could be pursued in future work.  Refinements are possible in both the mosquito-biasing algorithm and the robot flight simulation.  The model may be expanded to three dimensions.  Additional search paths may be compared to the existing algorithms.  These and other considerations will make a more realistic model for future work.  Full instrumentation of the multi-copter drone will allow more extensive testing of the hardware design and will allow field tests of the simulation algorithms.

%%%%%%%%%%%%%%%%%%%%%%%%%%%%%%%%%%%%%%%%%%%%%%%%%%%%%%%%%%%%%%%%%%%%%%%%%%%%%%%%



%%%%%%%%%%%%%%%%%%%%%%%%%%%%%%%%%%%%%%%%%%%%%%%%%%%%%%%%%%%%%%%%%%%%%%%%%%%%%%%%



%%%%%%%%%%%%%%%%%%%%%%%%%%%%%%%%%%%%%%%%%%%%%%%%%%%%%%%%%%%%%%%%%%%%%%%%%%%%%%%%
%\section*{APPENDIX}


%\section*{ACKNOWLEDGMENT}
%
%The authors would like to thank the Harris County Public Health \& Environmental Services Mosquito Control Division for their assistance and advice.

\section*{ACKNOWLEDGMENT}
The authors acknowledge the helpful advice and feedback from Martin Reyna Nava, MS, Medical Entomologist and Technical Operations Manager and Mustapha Debboun, Ph.D, BCE, Director Mosquito Control Division, of the Harris County Public Health \& Environmental Services, Mosquito Control Division.




%%%%%%%%%%%%%%%%%%%%%%%%%%%%%%%%%%%%%%%%%%%%%%%%%%%%%%%%%%%%%%%%%%%%%%%%%%%%%%%%


\bibliographystyle{IEEEtran}
\bibliography{./bib/mosquitorefs}%

%\begin{thebibliography}{99}
%
%\bibitem{c1} D. V. Maliti, N. J. Govella, G. F. Killeen, N. Mirzai, P. C. D. Johnson Development and evaluation of mosquito-electrocuting traps as alternatives to the human landing catch technique for sampling host-seeking malaria vectors, Malaria Journal, vol. 14:502, Dec. 2015
%\bibitem{c2} Anupa Elizabeth, P.; Saravana Mohan, M.; Philip Samuel, P.; Pandian, S.R.; Tyagi, B.K.,Identification and eradication of mosquito breeding sites using wireless networking and electromechanical technologies, in Recent Trends in Information Technology (ICRTIT), 2014 International Conference, Chennai, 2014, pp. 1-6.
%\bibitem{c3} Hur, B.; Eisenstadt, W., Low-power wireless climate monitoring system with RFID security access feature for mosquito and pathogen research, in Mobile and Secure Services (MOBISECSERV), 2015 First Conference, Gainsville, pp.1-5, 20-21 Feb. 2015
%
%
%
%
%
%\end{thebibliography}




\end{document}
